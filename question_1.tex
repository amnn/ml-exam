\subsection{Part (a)}

Suppose we follow our colleague's advice: After training the classifier using a soft-margin SVM on $\langle(\mathbf{x}_i,y_i)\rangle_{i=1}^m$ to get parameters $\mathbf{w}, w_0$, we try to find the $\alpha$ that minimises false-positive error on a fresh sample $\langle(\mathbf{v}_i,u_i)\rangle_{i=1}^k$.

Our colleague refers to this as cross-validation. For this to be true, we would need to treat $\alpha$ as a hyper-parameter, and for each value of $\alpha$, train a new classifier, checking its false-positive rate on the validation set, picking the $\alpha$ that yields the lowest such rate. This seems at odds with their claim that we do not need to retrain the classifier, but in fact it is not: The value of $\alpha$ does not affect the classifier that is learned (as it does not appear in the SVM objective or constraints), so we may train it once and then proceed to vary $\alpha$.

Given the optimal $\mathbf{w}^\ast$ and $w_0^\ast$ from training, our goal can be phrased in terms of finding the $\alpha^\ast$ that optimises program $V_1$ defined below (whose objective is exactly the number of false positive examples in the validation set).
\begin{align*}
  V_1: && \min_{\alpha} & \sum_{\substack{i=1\\u_i=-1}}^k\sembrack{\mathbf{v}_i\cdot\mathbf{w}^\ast+w_0^\ast\geq\alpha} &&
  \\   && \text{s.t. } & \alpha > 0 &&
\end{align*}
However, it is always possible to achieve an optimal value of $0$ for $V_1$ by setting:
\begin{align*}
  \alpha > \max(\{0\}\cup\{\mathbf{v}_i\cdot\mathbf{w}+w_0:1\leq i\leq k,~u_i=-1\})
\end{align*}
The consequence of this is that when optimising $V_1$, $\alpha$ will be sent to $\infty$, and our classification for a new email with features $\mathbf{x}$ becomes:
\begin{align*}
  \widehat{y} & =
  \begin{cases}
    1 & \text{if }\mathbf{x}\cdot\mathbf{w} + w_0\geq\infty\\
    -1 & \text{otherwise}
  \end{cases}
  \\ & = -1
\end{align*}
i.e. all emails are labelled as not spam!

In summary this method is poor because it does not take false-negative error into account at all.

\subsection{Part (b)}

To see whether this colleague's suggestion holds water let us look at its effect on the SVM optimisation problem. We will focus on the non-separable case because duplicating samples has no effect when the data is separable.

Suppose we are given $\mathbf{X}=\langle(\mathbf{x}_i,y_i)\rangle_{i=1}^m$ as our training set. The original (soft margin) optimisation problem was:

\begin{alignat*}{3}
  \min_{\mathbf{w},~w_0,~\boldsymbol\xi}\quad&&\frac{1}{2}\norm{\mathbf{w}}^2 + C\sum_{i=1}^m\xi_i\\
  \text{s.t. }\quad&&y_i(\mathbf{w}\cdot\mathbf{x}_i+w_0) & \geq 1 - \xi_i &&\quad\text{for }i = 1,\ldots,m
  \\ &&\xi_i & \geq 0 &&\quad\text{for }i=1,\ldots,m
\end{alignat*}

\noindent
Now let us partition $\mathbf{X}$ into
\begin{align*}
  \mathbf{X}^+ & = \langle(x_i^+,+1)\rangle_{i=1}^{m^+} && \text{where}\quad m^+ + m^- = m\\
  \mathbf{X}^- & = \langle(x_i^-,-1)\rangle_{i=1}^{m^-}
\intertext{And produce our new dataset by duplicating $X^+$ (The spam emails) 200 times:}
\overset{\sim}{X} & = X^-\concat\underbrace{X^+\concat\cdots\concat X^+}_{\text{200 times}} && \text{where}\quad\overset{\sim}{m} = m^-+200m^+
\end{align*}
\noindent And formulate the optimisation problem for this new data-set:
\begin{alignat*}{3}
  \min_{\mathbf{w},~w_0,~\boldsymbol\xi}\quad&&\frac{1}{2}\norm{\mathbf{w}}^2 + C\sum_{i=1}^m\xi_i\\
  \text{s.t. }\quad&&y_i(\mathbf{w}\cdot\mathbf{\overset{\sim}{x}}_i+w_0) & \geq 1 - \xi_i &&\quad\text{for }i = 1,\ldots,\overset{\sim}{m}
  \\ &&\xi_i & \geq 0 &&\quad\text{for }i=1,\ldots,\overset{\sim}{m}
  \intertext{But we may rearrange this program to group duplicated data together:}
  \min_{\mathbf{w},~w_0,~\boldsymbol\xi}\quad&&\frac{1}{2}\norm{\mathbf{w}}^2 + C\sum_{i=1}^m\xi_i\\
  \text{s.t. }\quad&&-1(\mathbf{w}\cdot\mathbf{x}_i^-+w_0) & \geq 1 - \xi_i &&\quad\text{for }i = 1,\ldots,m^-
  \\&&+1(\mathbf{w}\cdot\mathbf{x}_i^++w_0) & \geq 1 - \xi_j &&\quad\text{for }i = 1,\ldots,m^+
  \\&&&&&\quad\makebox[1.4em]{}k = 0,\ldots,199
  \\&&&&&\quad\makebox[1.6em]{\text{let }}j = m^- + km^++i
  \\ &&\xi_i & \geq 0 &&\quad\text{for }i=1,\ldots,\overset{\sim}{m}
\end{alignat*}
After rearranging, we notice that for a given $i=1,\ldots,m^+$, at the optimal solution, for all $0\leq k,l<200$, $\xi_a = \xi_b$ where $a = m^-+km^++i$ and $b = m^-+lm^++i$.

\begin{proof}
  To see why, suppose for a contradiction that $\xi^\ast_a<\xi^\ast_b$, in our optimal solution $(\mathbf{w}^\ast,~w_0^\ast,~\boldsymbol\xi^\ast)$.
  \begin{itemize}
    \step[\imps] $\mathbf{w}^\ast\cdot\mathbf{x}_i^+ + w_0^\ast\geq 1-\xi_a^\ast > 1-\xi_b^\ast$
    \step[\imps] There is slack in the constraint for $\xi_b^\ast$.
    \step[\imps] The objective could be reduced beyond that of the optimal solution by $C(\xi_b^\ast - \xi_a^\ast)$, by setting $\xi_b^\ast\coloneqq\xi_a^\ast$.
    \step[\contras] The proposed solution is not optimal
    \step[\imps] such an $\xi_a^\ast$, $\xi_b^\ast$ cannot exist.\qedhere
  \end{itemize}
\end{proof}

Given this equality between slack variables, we may replace the 200 slack variables per positive example with one, without any loss of freedom at the optimum:

\begin{alignat*}{3}
  \min_{\mathbf{w},~w_0,~\boldsymbol\xi^+,\boldsymbol\xi^-}\quad&&\frac{1}{2}\norm{\mathbf{w}}^2 + C(\sum_{i=1}^{m^-}\xi_i^-+200\sum_{i=1}^{m^+}\xi_i^+)\\
  \text{s.t. }\quad&&-1(\mathbf{w}\cdot\mathbf{x}_i^-+w_0) & \geq 1 - \xi_i^- &&\quad\text{for }i = 1,\ldots,m^-
  \\ &&+1(\mathbf{w}\cdot\mathbf{x}_i^++w_0) & \geq 1 - \xi_i^+ &&\quad\text{for }i = 1,\ldots,m^+
  \\ &&\xi_i^- & \geq 0 &&\quad\text{for }i=1,\ldots,m^-
  \\ &&\xi_i^+ & \geq 0 &&\quad\text{for }i=1,\ldots,m^+
\end{alignat*}

At this point it is worth observing that this has the effect of penalising false-\textit{negative} errors, not false-positive errors as we had intended. We can change this by moving the factor of 200 to the slack variables for the non-spam emails. In our original formulation, this would be equivalent to duplicating all the legitimate emails 200 times:

\begin{alignat*}{3}
  \min_{\mathbf{w},~w_0,~\boldsymbol\xi^+,\boldsymbol\xi^-}\quad&&\frac{1}{2}\norm{\mathbf{w}}^2 + C(200\sum_{i=1}^{m^-}\xi_i^-+\sum_{i=1}^{m^+}\xi_i^+)\\
  \text{s.t. }\quad&&-1(\mathbf{w}\cdot\mathbf{x}_i^-+w_0) & \geq 1 - \xi_i^- &&\quad\text{for }i = 1,\ldots,m^-
  \\ &&+1(\mathbf{w}\cdot\mathbf{x}_i^++w_0) & \geq 1 - \xi_i^+ &&\quad\text{for }i = 1,\ldots,m^+
  \\ &&\xi_i^- & \geq 0 &&\quad\text{for }i=1,\ldots,m^-
  \\ &&\xi_i^+ & \geq 0 &&\quad\text{for }i=1,\ldots,m^+
\end{alignat*}
\noindent And then the dual takes the form of:
\begin{align*}
  \max_{\boldsymbol\alpha}\quad&\sum_{i}^m\alpha_i-\frac{1}{2}\sum_{i,j}^{m}\alpha_i\alpha_jy_iy_j\mathbf{x}_i\cdot\mathbf{x}_j\\
  \text{s.t.}\quad&\sum_i^my_i\alpha_i = 0
  \\ & 0\leq\alpha_i\leq C && \text{for }i=1,\ldots,m\text{ where }y_i=+1
  \\ & 0\leq\alpha_i\leq 200C &&\text{for }i=1,\ldots,m\text{ where }y_i=-1
\end{align*}

The original formulation would have been less efficient (in time \textit{and} space consumption) than a standard soft margin classifier. In its dual form, it would have had a new variable for each duplicated email, and if we were to kernelise it, the number of legitimate emails used as support vectors could have (in the worst case) gone up by a factor of 200 as well.

The equivalent formulation which changes the cost of slack, however, has the same number of variables in the dual form as the standard soft margin classifier, so it allows us to use the idea of duplication with no performance penalty.

\subsection{Part (c)}
We use a similar idea to that of Part (b), but instead of simplying paying \textit{more} for false-positive slack, we make it infinitely expensive to have false-positive slack. This can be interpreted as having the margin facing legitimate emails being \textbf{hard} and the margin facing spam emails \textbf{soft}.\\[1em]

\noindent\textbf{Primal}
\begin{alignat*}{3}
  \min_{\mathbf{w},~w_0,~\boldsymbol\xi^+}\quad&&\frac{1}{2}\norm{\mathbf{w}}^2 + C\sum_{i=1}^{m^+}\xi_i^+\\
  \text{s.t. }\quad&&-1(\mathbf{w}\cdot\mathbf{x}_i^-+w_0) & \geq 1 &&\quad\text{for }i = 1,\ldots,m^-
  \\ &&+1(\mathbf{w}\cdot\mathbf{x}_i^++w_0) & \geq 1 - \xi_i^+ &&\quad\text{for }i = 1,\ldots,m^+
  \\ &&\xi_i^+ & \geq 0 &&\quad\text{for }i=1,\ldots,m^+
\end{alignat*}

\noindent\textbf{Dual}
\begin{align*}
  \max_{\boldsymbol\alpha}\quad&\sum_{i}^m\alpha_i-\frac{1}{2}\sum_{i,j}^{m}\alpha_i\alpha_jy_iy_j\mathbf{x}_i\cdot\mathbf{x}_j\\
  \text{s.t.}\quad&\sum_i^my_i\alpha_i = 0
  \\ & 0\leq\alpha_i\leq C && \text{for }i=1,\ldots,m\text{ where }y_i=+1
  \\ & 0\leq\alpha_i &&\text{for }i=1,\ldots,m\text{ where }y_i=-1
\end{align*}

\begin{prop}
  The optimal classifier $\mathbf{w},~w_0$ correctly classifies all legitimate emails in the training set.
\end{prop}
\begin{proof}
  Observe that as $\mathbf{w},~w_0$ is an optimal classifier, it is feasible in the primal, as a result, for all negative samples $\mathbf{x}_i^-$ (corresponding to non-spam/legitimate emails) we have that:

  \begin{flalign*}
    &-1(\mathbf{w}\cdot\mathbf{x}_i^-+w_0)\geq 1 &&
    \\\iffs\quad& \mathbf{w}\cdot\mathbf{x}_i^-+w_0\leq -1 &&
    \\\imps\quad& \text{The classifier classes  $\mathbf{x}_i^-$ as $-1$ (not spam)} &&
    \tag*{\qedhere}
  \end{flalign*}
\end{proof}

In order to minimise false-negative errors, we should perform a cross-validated grid-search to choose an appropriate value of $C$ that minimises validation error.
