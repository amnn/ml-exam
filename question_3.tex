\subsection{Part (a)}

Degradation, as it pertains to feed forward neural networks in [***CITATION***] refers to the phenomenon whereby as the depth of the network increases, both the training and test accuracy saturate (change very little with increasing depth), and then start to get worse.\\[1em]

This runs counter to expectation: As the depth of a network increases, the number of parameters in the model also increases, and typically, this cases a model to \textbf{overfit}. This would be characterised by the training accuracy constantly improving with depth until it reaches 100\% whilst after a point, the test accuracy starts to suffer. This is different from degradation in which both accuracies suffer.\\[1em]

In the case of linear regression with a polynomial basis expansion, I would expect overfit to occur, and not degradation: It is always possible to fit a polynomial of degree $\geq n$ through $n$ points, and linear regression will always find such a fit (if regularisation techniques are not employed), so given $n$ sample points, as the degree $d$ of the basis expansion approaches $n$ from below, I would expect training accuracy to approach 100\% (and stay at 100\% as $d$ exceeds $n$). Meanwhile, test accuracy will deterioriate as this model will generalise poorly.\\[1em]

The reason positted in [***CITATION***] for feed-forward neural networks' susceptibility to degradation is that optimisation techniques may have trouble approximating identity mappings using multiple layers of non-linearities. Such a difficulty would mean that even if there is a $d$-layer neywork, $N$, with better training accuracy than a given $d+k$ layer network ($k > 0$), $N^\prime$, by optimising the latter, we are unlikely to reach a network where the first $d$ layers resemble the former, and the last $k$ layers are identity mappings (even though this is an improvement), or indeed any other interleaving of layers in $N$ and identity mappings.

\subsection{Part (b)}

Both Residual [***CITATION***] and Highway [***CITATION***] networks address issues that arise when attempting to train very deep feed-forward networks, by designing networks where the input to a layer can skip it (via a shortcircuit, or an ``information highway'' respectively) and be sent through to later layers.\\[1em]

Given a plain 4-layer network with $\operatorname{ReLU}$ non-linearities (biases omitted for brevity):
\begin{align*}
  &\operatorname{ReLU} : \mathbb{R}^n\to\mathbb{R}^n\\
  &\operatorname{ReLU}(\mathbf{x}) = \mathbf{x}^\prime\text{ where }x^\prime_i = \max(0,x_i)
\end{align*}
\begin{align*}
  &\text{Weights}\quad&\mathbf{W}_i & \in\mathbb{R}^{n\times n} && \text{for}\quad i = 1,\ldots,4\\
  &&\mathbf{z}_i & \in\mathbb{R} &&\text{for}\quad i = 0,\ldots,4\\
  &&\mathbf{z}_{i+1} & = \operatorname{ReLU}(\mathbf{W}_{i+1}\mathbf{z}_i) &&\text{for}\quad i = 0,\ldots,3\\
  &\textbf{input}\quad&\mathbf{x} & = \mathbf{z}_0\\
  &\textbf{output}\quad&\mathbf{y} & = \mathbf{z}_4
\end{align*}

We may make it a residual network by adding shortcircuits every 2 layers:\marginnote{The choice of 2 layers was arbitrary for the sake of the example, both techniques work when skipping any number of layers, although they must skip over a non-linearity in order to change the behaviour of the network.}

\begin{align*}
  \mathbf{z}_{i+1} & =
  \begin{cases}
    \operatorname{ReLU}(\mathbf{W}_{i+1}\mathbf{z}_i + \mathbf{z}_{i-1})
    & \text{ if $i+1$ is even}\\
    \operatorname{ReLU}(\mathbf{W}_{i+1}\mathbf{z}_i)
    & \text{ otherwise}
  \end{cases}
\end{align*}

Or we may make it a highway network by adding a transfer and carry gate after every 2 layers:

\begin{align*}
  \mathbf{z}_{i+1} & =
  \begin{cases}
    \operatorname{ReLU}(\mathbf{W}_{i+1}\mathbf{z}_i)\odot T(\mathbf{z}_{i-1},\mathbf{U}_{i+1})
    +\mathbf{z}_{i-1}\odot C(\mathbf{z}_{i-1},\mathbf{V}_{i+1})
    & \text{ if $i+1$ is even}\\
    \operatorname{ReLU}(\mathbf{W}_{i+1}\mathbf{z}_i)
    & \text{ otherwise}
  \end{cases}
\end{align*}

Where $T$ and $C$ define the transform and carry gates respectively and $\odot$ denotes elementwise multiplication.\\[1em]

As can be seen, the ``skipping'' is not true skipping in either case. Instead, the input to a layer is mixed with the output of a later layer.\\[1em]

Residual networks could be considered a special case of Highway networks, where the transfer and carry gates are both always open. As a result, short-circuits in Residual networks add no extra parameters whereas those in Highway networks do ($\mathbf{U}$ and $\mathbf{V}$ in the example).\\[1em]

\subsection{Part (c)}

Lesioning is the process of essentially ``disabling'' a layer: Its transfer gate is manually set to $\mathbf{0}$ and (as the carry gate is defined by $\mathbf{1}-\mathbf{T}$ in the paper) its carry gate becomese $\mathbf{1}$, causing it to just copy its input to its output.

It was used to see which layers are significant to the output of the network. The conclusion was that in most networks, lesioning early layers was most detrimental, with later layers becoming progressively less significant. What is more, for networks for simpler learning problems, like MNIST (Handwritten digits), later layers contributed almost nothing to the output, but an equally deep highway network used for a more complex learning task, such as CIFAR-100 (Object recognition, 100 classes), utilised all of its layers. The method used to determine this was, after training, a given layer was lesioned, the network was run on the training set again and the difference in the error was measured. A large difference suggested the layer contributed to the network, and a small difference in error suggested the opposite.\\[1em]

This indicates that Highway networks are capable of scaling he number of layers it uses based on the needs of the specific classification task at hand, which is a desirable trait but one that (as hinted in [***CITATION***]) is not easy to achieve with plain feed forward networks.
