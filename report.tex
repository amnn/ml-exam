\documentclass{tufte-handout}

\title{Machine Learning 2016}
\author{Candidate No. 683444}

\usepackage[ruled,vlined]{algorithm2e}
\usepackage{amsmath}
\usepackage{amssymb}
\usepackage{amsthm}
\usepackage{bm}
\usepackage{bussproofs}
\usepackage{calc}
\usepackage{enumerate}
\usepackage{mathtools}
\usepackage{relsize}
\usepackage{stmaryrd}
\usepackage{tikz}
\usepackage{wasysym}

\usetikzlibrary{cd}

%%% Title Formatting

\titleformat*{\section}{\normalfont\Large\scshape}
\titleformat*{\subsection}{\normalfont\large\scshape}
\newcommand{\sectionbreak}{\clearpage}

%%% Custom Commands

\newcommand{\Lemma}{\textbf{lemma}}
\newcommand{\Thm}{\textbf{thm}}
\newcommand{\Def}{\textbf{def}}
\newcommand{\Contra}{\textbf{contra}}
\newcommand{\Assoc}{\textbf{assoc}}
\newcommand{\Trans}{\textbf{trans}}
\newcommand{\Refl}{\textbf{refl}}
\newcommand{\Hyp}{\textbf{hyp}}
\newcommand{\Ass}{\textbf{assume}}
\newcommand{\Intro}{\textbf{intro}}
\newcommand{\Elim}{\textbf{elim}}
\newcommand{\Let}{\textbf{let}}
\newcommand{\Epic}{\textbf{epic}}
\newcommand{\Monic}{\textbf{monic}}
\newcommand{\compose}{\circ}
\newcommand{\epic}{\twoheadrightarrow}
\newcommand{\cat}[1]{\mathcal{#1}}
\newcommand{\opcat}[1]{\mathcal{#1}^{\text{op}}}
\renewcommand{\hom}[3]{\cat{#1}(#2, #3)}
\newcommand{\Exp}{\Rightarrow}
\newcommand{\sembrack}[1]{\llbracket #1 \rrbracket}

\newcommand{\step}[1][\phantom{=}]{\item[{\makebox[{\widthof{$\Leftrightarrow$}}][l]{$#1$}}]}
\newcommand{\subp}[1]{\item[$#1$]}
\newcommand{\iffs}{\Leftrightarrow}
\newcommand{\imps}{\Rightarrow}
\newcommand{\contras}{\divideontimes}

\def\mathnote#1{%
  \tag*{\rlap{\hspace\marginparsep\smash{\parbox[t]{\marginparwidth}{%
  \footnotesize#1}}}}
}

\DeclarePairedDelimiter\abs{\lvert}{\rvert}
\DeclareMathOperator{\id}{id}
\DeclareMathOperator{\FV}{FV}
\DeclareMathOperator{\App}{App}
\DeclareMathOperator{\Widen}{\nabla}
\DeclareMathOperator{\Narrow}{\triangle}
\DeclareMathOperator{\Infty}{\mathit{infty}}

%%% Theorem styles

\theoremstyle{definition}
\newtheorem{definition}{Definition}
\numberwithin{definition}{section}

\theoremstyle{plain}
\newtheorem{prop}{Proposition}
\numberwithin{prop}{section}

\theoremstyle{plain}
\newtheorem{lemma}{Lemma}
\numberwithin{lemma}{section}

\theoremstyle{plain}
\newtheorem{corollary}{Corollary}
\numberwithin{corollary}{section}

%%% Initialise Counters

\setcounter{section}{1}
\maxdeadcycles=1000

%%% Proof Trees
\EnableBpAbbreviations

%%% Content
\begin{document}
\maketitle

\section{Question 1}\label{sec:q-1}
\subsection{Part (a)}

Suppose we follow our colleague's advice: After training the classifier using a soft-margin SVM on $\langle(\mathbf{x}_i,y_i)\rangle_{i=1}^m$ to get parameters $\mathbf{w}, w_0$, we try to find the $\alpha$ that minimises false-positive error on a fresh sample $\langle(\mathbf{v}_i,u_i)\rangle_{i=1}^k$.\\[1em]

Our colleague refers to this as cross-validation. For this to be true, we would need to treat $\alpha$ as a hyper-parameter, and for each value of $\alpha$, train a new classifier, checking its false-positive rate on the validation set, picking the $\alpha$ that yields the lowest such rate. This seems at odds with their claim that we do not need to retrain the classifier, but in fact it is not: The value of $\alpha$ does not affect the classifier that is learned (as it does not appear in the SVM objective or constraints), so we may train it once and then proceed to vary $\alpha$.\\[1em]

Given the optimal $\mathbf{w}^\ast$ and $w_0^\ast$ from training, our goal can be phrased in terms of finding the $\alpha^\ast$ that optimises program $V_1$ defined below (whose objective is exactly the number of false positive examples in the validation set).
\begin{align*}
  V_1: && \min_{\alpha} & \sum_{\substack{i=1\\u_i=-1}}^k\sembrack{\mathbf{v}_i\cdot\mathbf{w}^\ast+w_0^\ast\geq\alpha} &&
  \\   && \text{s.t. } & \alpha > 0 &&
\end{align*}
However, it is always possible to achieve an optimal value of $0$ for $V_1$ by setting:
\begin{align*}
  \alpha > \max(\{0\}\cup\{\mathbf{v}_i\cdot\mathbf{w}+w_0:1\leq i\leq k,~u_i=-1\})
\end{align*}
The consequence of this is that when optimising $V_1$, $\alpha$ will be sent to $\infty$, and our classification for a new email with features $\mathbf{x}$ becomes:
\begin{align*}
  \widehat{y} & =
  \begin{cases}
    1 & \text{if }\mathbf{x}\cdot\mathbf{w} + w_0\geq\infty\\
    -1 & \text{otherwise}
  \end{cases}
  \\ & = -1
\end{align*}
i.e. all emails are labelled as not spam!\\[1em]

In summary this method is poor because it does not take false-negative error into account at all.

\subsection{Part (b)}

To see whether this colleague's suggestion holds water let us look at its effect on the SVM optimisation problem. We will focus on the non-separable case because duplicating samples has no effect when the data is separable.\\[1em]

Suppose we are given $\mathbf{X}=\langle(\mathbf{x}_i,y_i)\rangle_{i=1}^m$ as our training set. The original (soft margin) optimisation problem was:

\begin{alignat*}{3}
  \min_{\mathbf{w},~w_0,~\boldsymbol\xi}\quad&&\frac{1}{2}\norm{\mathbf{w}}^2 + C\sum_{i=1}^m\xi_i\\
  \text{s.t. }\quad&&y_i(\mathbf{w}\cdot\mathbf{x}_i+w_0) & \geq 1 - \xi_i &&\quad\text{for }i = 1,\ldots,m
  \\ &&\xi_i & \geq 0 &&\quad\text{for }i=1,\ldots,m
\end{alignat*}

\noindent
Now let us partition $\mathbf{X}$ into
\begin{align*}
  \mathbf{X}^+ & = \langle(x_i^+,+1)\rangle_{i=1}^{m^+} && \text{where}\quad m^+ + m^- = m\\
  \mathbf{X}^- & = \langle(x_i^-,-1)\rangle_{i=1}^{m^-}
\intertext{And produce our new dataset by duplicating $X^+$ (The spam emails) 200 times:}
\overset{\sim}{X} & = X^-\concat\underbrace{X^+\concat\cdots\concat X^+}_{\text{200 times}} && \text{where}\quad\overset{\sim}{m} = m^-+200m^+
\end{align*}
\noindent And formulate the optimisation problem for this new data-set:
\begin{alignat*}{3}
  \min_{\mathbf{w},~w_0,~\boldsymbol\xi}\quad&&\frac{1}{2}\norm{\mathbf{w}}^2 + C\sum_{i=1}^m\xi_i\\
  \text{s.t. }\quad&&y_i(\mathbf{w}\cdot\mathbf{\overset{\sim}{x}}_i+w_0) & \geq 1 - \xi_i &&\quad\text{for }i = 1,\ldots,\overset{\sim}{m}
  \\ &&\xi_i & \geq 0 &&\quad\text{for }i=1,\ldots,\overset{\sim}{m}
  \intertext{But we may rearrange this program to group duplicated data together:}
  \min_{\mathbf{w},~w_0,~\boldsymbol\xi}\quad&&\frac{1}{2}\norm{\mathbf{w}}^2 + C\sum_{i=1}^m\xi_i\\
  \text{s.t. }\quad&&-1(\mathbf{w}\cdot\mathbf{x}_i^-+w_0) & \geq 1 - \xi_i &&\quad\text{for }i = 1,\ldots,m^-
  \\&&+1(\mathbf{w}\cdot\mathbf{x}_i^++w_0) & \geq 1 - \xi_j &&\quad\text{for }i = 1,\ldots,m^+
  \\&&&&&\quad\makebox[1.4em]{}k = 0,\ldots,199
  \\&&&&&\quad\makebox[1.5em]{\text{let }}j = m^- + km^++i
  \\ &&\xi_i & \geq 0 &&\quad\text{for }i=1,\ldots,\overset{\sim}{m}
\end{alignat*}
After rearranging, we notice that for a given $i=1,\ldots,m^+$, at the optimal solution, for all $0\leq k,l<200$, $\xi_a = \xi_b$ where $a = m^-+km^++i$ and $b = m^-+lm^++i$.

\begin{proof}
  To see why, suppose for a contradiction that $\xi^\ast_a<\xi^\ast_b$, in our optimal solution $(\mathbf{w}^\ast,~w_0^\ast,~\boldsymbol\xi^\ast)$.
  \begin{itemize}
    \step[\imps] $\mathbf{w}^\ast\cdot\mathbf{x}_i^+ + w_0^\ast\geq 1-\xi_a^\ast > 1-\xi_b^\ast$
    \step[\imps] There is slack in the constraint for $\xi_b^\ast$.
    \step[\imps] The objective could be reduced beyond that of the optimal solution by $C(\xi_b^\ast - \xi_a^\ast)$, by setting $\xi_b^\ast\coloneqq\xi_a^\ast$.
    \step[\contras] The proposed solution is not optimal
    \step[\imps] such an $\xi_a^\ast$, $\xi_b^\ast$ cannot exist.\qedhere
  \end{itemize}
\end{proof}

Given this equality between slack variables, we may replace the 200 slack variables per positive example with one, without any loss of freedom at the optimum:

\begin{alignat*}{3}
  \min_{\mathbf{w},~w_0,~\boldsymbol\xi^+,\boldsymbol\xi^-}\quad&&\frac{1}{2}\norm{\mathbf{w}}^2 + C(\sum_{i=1}^{m^-}\xi_i^-+200\sum_{i=1}^{m^+}\xi_i^+)\\
  \text{s.t. }\quad&&-1(\mathbf{w}\cdot\mathbf{x}_i^-+w_0) & \geq 1 - \xi_i^- &&\quad\text{for }i = 1,\ldots,m^-
  \\ &&+1(\mathbf{w}\cdot\mathbf{x}_i^++w_0) & \geq 1 - \xi_i^+ &&\quad\text{for }i = 1,\ldots,m^+
  \\ &&\xi_i^- & \geq 0 &&\quad\text{for }i=1,\ldots,m^-
  \\ &&\xi_i^+ & \geq 0 &&\quad\text{for }i=1,\ldots,m^+
\end{alignat*}

At this point it is worth observing that this has the effect of penalising false-\textit{negative} errors, not false-positive errors as we had intended. We can change this by moving the factor of 200 to the slack variables for the non-spam emails. In our original formulation, this would be equivalent to \textbf{duplicating all the legitimate emails} 200 times:

\begin{alignat*}{3}
  \min_{\mathbf{w},~w_0,~\boldsymbol\xi^+,\boldsymbol\xi^-}\quad&&\frac{1}{2}\norm{\mathbf{w}}^2 + C(200\sum_{i=1}^{m^-}\xi_i^-+\sum_{i=1}^{m^+}\xi_i^+)\\
  \text{s.t. }\quad&&-1(\mathbf{w}\cdot\mathbf{x}_i^-+w_0) & \geq 1 - \xi_i^- &&\quad\text{for }i = 1,\ldots,m^-
  \\ &&+1(\mathbf{w}\cdot\mathbf{x}_i^++w_0) & \geq 1 - \xi_i^+ &&\quad\text{for }i = 1,\ldots,m^+
  \\ &&\xi_i^- & \geq 0 &&\quad\text{for }i=1,\ldots,m^-
  \\ &&\xi_i^+ & \geq 0 &&\quad\text{for }i=1,\ldots,m^+
\end{alignat*}
\noindent And then the dual takes the form of:
\begin{align*}
  \max_{\boldsymbol\alpha}\quad&\sum_{i}^m\alpha_i-\frac{1}{2}\sum_{i,j}^{m}\alpha_i\alpha_jy_iy_j\mathbf{x}_i\cdot\mathbf{x}_j\\
  \text{s.t.}\quad&\sum_i^my_i\alpha_i = 0
  \\ & 0\leq\alpha_i\leq C && \text{for }i=1,\ldots,m\text{ where }y_i=+1
  \\ & 0\leq\alpha_i\leq 200C &&\text{for }i=1,\ldots,m\text{ where }y_i=-1
\end{align*}

The original formulation would have been less efficient (in time \textit{and} space consumption) than a standard soft margin classifier. In its dual form, it would have had a new variable for each duplicated email, and if we were to kernelise it, the number of legitimate emails used as support vectors could have (in the worst case) gone up by a factor of 200 as well.\\[1em]

The equivalent formulation which changes the cost of slack, however, has the same number of variables in the dual form as the standard soft margin classifier, so it allows us to use the idea of duplication with no performance penalty.

\subsection{Part (c)}
We use a similar idea to that of Part (b), but instead of simply paying \textit{more} for false-positive slack, we make it infinitely expensive to have false-positive slack. This can be interpreted as having the margin facing legitimate emails being \textbf{hard} and the margin facing spam emails \textbf{soft}.\\[1em]

\noindent\textbf{Primal}
\begin{alignat*}{3}
  \min_{\mathbf{w},~w_0,~\boldsymbol\xi^+}\quad&&\frac{1}{2}\norm{\mathbf{w}}^2 + C\sum_{i=1}^{m^+}\xi_i^+\\
  \text{s.t. }\quad&&-1(\mathbf{w}\cdot\mathbf{x}_i^-+w_0) & \geq 1 &&\quad\text{for }i = 1,\ldots,m^-
  \\ &&+1(\mathbf{w}\cdot\mathbf{x}_i^++w_0) & \geq 1 - \xi_i^+ &&\quad\text{for }i = 1,\ldots,m^+
  \\ &&\xi_i^+ & \geq 0 &&\quad\text{for }i=1,\ldots,m^+
\end{alignat*}

\noindent\textbf{Dual}
\begin{align*}
  \max_{\boldsymbol\alpha}\quad&\sum_{i}^m\alpha_i-\frac{1}{2}\sum_{i,j}^{m}\alpha_i\alpha_jy_iy_j\mathbf{x}_i\cdot\mathbf{x}_j\\
  \text{s.t.}\quad&\sum_i^my_i\alpha_i = 0
  \\ & 0\leq\alpha_i\leq C && \text{for }i=1,\ldots,m\text{ where }y_i=+1
  \\ & 0\leq\alpha_i &&\text{for }i=1,\ldots,m\text{ where }y_i=-1
\end{align*}

\begin{prop}
  The optimal classifier $\mathbf{w},~w_0$ correctly classifies all legitimate emails in the training set.
\end{prop}
\begin{proof}
  Observe that as $\mathbf{w},~w_0$ is an optimal classifier, it is feasible in the primal, as a result, for all negative samples $\mathbf{x}_i^-$ (corresponding to non-spam/legitimate emails) we have that:

  \begin{flalign*}
    &-1(\mathbf{w}\cdot\mathbf{x}_i^-+w_0)\geq 1 &&
    \\\iffs\quad& \mathbf{w}\cdot\mathbf{x}_i^-+w_0\leq -1 &&
    \\\imps\quad& \text{The classifier classes  $\mathbf{x}_i^-$ as $-1$ (not spam)} &&
    \tag*{\qedhere}
  \end{flalign*}
\end{proof}

In order to minimise false-negative errors, we should perform a cross-validated grid-search to choose an appropriate value of $C$ that minimises validation error.


\section{Question 2}\label{sec:q-2}
\subsection{Part (a)}

Our model treats the each distinct input as an independent Bernoulli random variable, whose probability of success is governed by the input triple that was posed as the question, so we have as many free parameters as the size of our input space. To determine this size we note that, given an input triple $(a : b, c)$:
\begin{itemize}[-]
\itemsep0em
\item All fruit are distinct, that is to say $a\neq b$, $b\neq c$ and $a\neq c$.
\item The choices are ordered: $b\prec c$.
\end{itemize}

There are $n(n-1)(n-2)$ triples of distinct fruit, however, in half of them, the order of the second and third fruit is inverted, so in actuality there are:
\begin{align*}
  \frac{n(n-1)(n-2)}{2}\quad\text{parameters}
\end{align*}

Because we treat each input independently, we may say that ${y_i\sim\mathbb{1}(p_{b_ic_i}^{a_i})}$. As such, we may give the likelihood as below (taking $\mathbf{p}$ to mean the parameters to our model):

\begin{itemize}
  \step $\operatorname{L}(\langle((a_i : b_i,c_i),y_i)\rangle_{i=1}^m\mid\mathbf{p})$
  \step[=] $\displaystyle\prod\limits_{\substack{i=1\\y_i=1}}^mp_{b_i,c_i}^{a_i}\cdot\prod_{\substack{i=1\\y_i=0}}^m(1 - p_{b_i,c_i}^{a_i})$
\end{itemize}

\subsection{Part (b)}

\begin{itemize}
  \step $\operatorname{NLL}(\langle((a_i:b_i,c_i),y_i)\rangle_{i=1}^m\mid\mathbf{M})$
  \step[=] $-\ln\operatorname{L}(\langle((a_i:b_i,c_i),y_i)\rangle_{i=1}^m\mid\mathbf{M})$
  \step[=] $\displaystyle-\ln\left[\prod_{\substack{i=1\\y_i=1}}^m\frac{e^{M_{a_i,b_i}}}{e^{M_{a_i,b_i}} + e^{M_{a_i,c_i}}}\cdot\prod_{\substack{i=1\\y_i=0}}^m\frac{e^{M_{a_i,c_i}}}{e^{M_{a_i,b_i}}+e^{M_{a_i,c_i}}}\right]$
  \marginnote{Definition of Likelihood (as above)}
  \step[=] $\displaystyle-\sum_{\substack{i=1\\y_i=1}}^m\left[M_{a_i,b_i} - \ln(e^{M_{a_i,b_i}}+e^{M_{a_i,c_i}})\right]$
  \marginnote{Distributing $\ln$ through $\prod$}
  \step $\displaystyle-\sum_{\substack{i=1\\y_i=0}}^m\left[M_{a_i,c_i} - \ln(e^{M_{a_i,b_i}}+e^{M_{a_i,c_i}})\right]$
  \step[=] $\displaystyle\sum_{i=1}^m\ln(e^{M_{a_i,b_i}} + e^{M_{a_i,c_i}}) - \sum_{\substack{i=1\\y_i=1}}^mM_{a_i,b_i} - \sum_{\substack{i=1\\y_i=0}}^mM_{a_i,c_i}$
\end{itemize}

So, when our goal is to find a maximum likelihood estimate, our optimisation problem becomes:
\begin{align*}
  \text{MLE:}\quad\max_{\mathbf{M}}\quad&&
  \sum_{i=1}^m\ln(e^{M_{a_i,b_i}} + e^{M_{a_i,c_i}}) & - \sum_{\substack{i=1\\y_i=1}}^mM_{a_i,b_i} - \sum_{\substack{i=1\\y_i=0}}^mM_{a_i,c_i}\\
  \text{s.t. }\quad&& M_{a,a} & = 1 && \text{for fruit }a
  \\ && M_{a,b} + M_{b,a} & = 1      && \text{for distinct fruits }a,~b
  \\ && 0 \leq M_{a,b} & \leq 1     && \text{for pairs of fruits }a,~b
\end{align*}

\begin{prop}
  MLE is a convex optimisation problem\\[1em]
  \noindent It suffices to show that
  \begin{enumerate}[1.]
    \item All inequality constraints are convex
    \item All equality constraints are linear
    \item The objective function is convex
  \end{enumerate}
\end{prop}

We first make some observations about convex functions, namely that:
\begin{enumerate}[(i)]
  \item Linear functions are convex.
  \item Convex functions are closed under addition.
\end{enumerate}

\begin{proof}[Proof (1/2)]
  By observation (i), as all the constraints are linear, conditions 1 and 2 are met.\qedhere
\end{proof}

\begin{proof}[Proof 3]
  It remains to show that the objective is also convex. First, we split it up as follows:
  \begin{align*}
    \operatorname{MLE}(\mathbf{M}) & =
    \sum_{i=1}^mF_i(\mathbf{M}) - G(\mathbf{M}) - H(\mathbf{M})
    \\\quad\text{where}\quad &
    \\F_i(\mathbf{M}) & = \ln(e^{M_{a_i,b_i}}+e^{M_{a_i,b_i}})
    \\G(\mathbf{M}) & = \sum_{i=1:~y_i=1}^mM_{a_i,b_i}
    \\H(\mathbf{M}) & = \sum_{i=1:~y_i=0}^mM_{a_i,c_i}
  \end{align*}

  $G$ and $H$ are linear, so by observation (i), are convex. Then, by the closure of convexity under addition, it suffices to show that $F_i$ is convex for any $i$. $F_i$ is the projection of $\mathbf{M}$ to $\mathbb{R}^2$ followed by the application of $f(x_1,x_2)=\ln(e^{x_1}+x^{x_2})$, so the convexity of the former follows from that of the latter.

  As $f$ is twice differentiable, we may show that $f$ is convex by verifying that $f$'s Hessian is positive semi-definite.
  \begin{align*}
    \nabla_{\mathbf{x}}f & = \frac{1}{e^{x_1}+e^{x_2}}
    \begin{bmatrix}
      e^{x_1}\\
      e^{x_2}
    \end{bmatrix}
    \\\frac{\partial^2 f}{\partial x_1^2}
    & = \frac{\partial}{\partial x_1}\frac{e^{x_1}}{e^{x_1}+e^{x_2}}
    & = \frac{e^{x_1}(e^{x_1}+e^{x_2}) - e^{2x_1}}{(e^{x_1}+e^{x_2})^2}
    & = \frac{e^{x_1+x_2}}{(e^{x_1}+e^{x_2})^2}
    \\\frac{\partial^2 f}{\partial x_1\partial x_2}
    & = \frac{\partial}{\partial x_1}\frac{e^{x_2}}{e^{x_1}+e^{x_2}}
    && = \frac{-e^{x_1+x_2}}{(e^{x_1}+e^{x_2})^2}
    \intertext{\ldots and symmetrically for \nth{2} differentials in $x_2$}
    \implies\nabla_{\mathbb{x}}^2f & = \frac{e^{x_1+x_2}}{(e^{x_1}+e^{x_2})^2}
    \begin{bmatrix}
      1 & -1\\
      -1 & 1
    \end{bmatrix}
  \end{align*}
  Let $\left[\begin{smallmatrix}a\\b\end{smallmatrix}\right]\in\mathbb{R}^2$
  \begin{align*}
    [\,a,~b\,]
    \nabla_{\mathbf{x}}^2f
    \begin{bmatrix}
      a\\b
    \end{bmatrix}
    & =
    \frac{e^{x_1+x_2}}{(e^{x_1}+e^{x_2})^2}\cdot
      [\,a,~b\,]
    \cdot
    \begin{bmatrix}
      a - b\\
      b - a
    \end{bmatrix}
    \\ & = \frac{e^{x_1+x_2}}{(e^{x_1}+e^{x_2})^2}(a(a-b) + b(b-a))
    \\ & = \frac{e^{x_1+x_2}}{(e^{x_1}+e^{x_2})^2}(a - b)^2
    \\ & \geq 0
    \mathnote{%
      \vspace{-2em}
      \begin{align*}
      (a-b)^2         & \geq 0 \text{ for any } & a, b & \in\mathbb{R} &&&&\\
      \text{and } e^x & \geq 0 \text{ for any } &    x & \in\mathbb{R} &&&&
      \end{align*}
    }
  \end{align*}
  \noindent $\implies \nabla_{\mathbf{x}}^2f$ is positive semi-definite.
  \\\noindent $\implies f$ is convex.\qedhere
\end{proof}

Although we have proven that \textit{MLE} is a convex program, we \textbf{cannot} use \textit{Interior Point Methods}, to find an optimum in polynomial time, as these techniques are only known to work for Linear Programs, Second-Order Cone Programs and Semidefinite Programs [***CITATION***], all of which have linear objective functions (which is not the case for \textit{MLE}).\\[1em]

However, convexity does tell us that it suffices for us to search for a local minimum (as that is guaranteed to also be the global minimum, although perhaps not a unique global minimum). To this end, we might turn to \textit{Higher-Order Newton-Raphson Iteration}. One down-side to this approach is that it converges on stationary points, not optima. Whilst we have shown that the objective is convex, we do not know that it is strictly convex, as such, it may have saddle points that Newton-Raphson could get stuck in. Another issue is that in order to run Newton-Raphson, at every iteration, we must calculate the gradient and the inverse Hessian for the current parameters, for large numbers of fruit, $n$, this could be a prohibitively expensive operation, as the gradient is a vector in $\mathbb{R}^{n^2}$ and the Hessian is a matrix in $\mathbb{R}^{n^2\times n^2}$.\\[1em]

Another viable approach is to try gradient descent, which is less prone to getting stuck in saddles, especially when techniques such as momentum are used. It is also generally more efficient than Newton-Raphson iteration, as we need only calculate the gradient of the objective with respect to one parameter at every iteration. \textit{Stochastic Gradient Descent}, and \textit{AdaGrad} are both good examples of this technique.\\[1em]

We may also wish to try \textit{Coordinate descent}, whereby we cycle through the parameters and at each iteration, vary the parameter of interest in order to reduce the objective, (this bears some similarity to SGD). This technique enjoys the same benefits as the aforementioned gradient descent methods, and like them, has been shown to be effective when used to optimise other convex optimisation problems, such as the dual (linear) SVM formulation [***CITATION***].

\subsection{Part (c)}

One option is to treat $\mathbf{M}$ as a similarity measure and perform \textit{heirarchical clustering}, measuring the similarity between clusters by the average pairwise similarity of their respective components. This idea of heirarchy may be a useful one as fruit (or the plants they come from) already fall into the taxonomic heirarchy.\\[1em]

As the question tells us that $\mathbf{M}$ is positive semidefinite, we may also embed $\mathbf{M}$ into Euclidean space. Such an embedding allows us to use other techniques, such as \textit{kernel principal component analysis}, \textit{k-means clustering} or \textit{spectral clustering} which may produce clusters with a better fit than heirarchical clustering.\\[1em]

One such embedding can be achieved with \textit{multidimensional scaling}: As we know that $\mathbb{M}$ is positive semidefinite, we know that it arises as the Gram matrix of some $n$ vectors in $\mathbb{R}^f$, let them be the rows of $\mathbf{X}\in\mathbb{R}^{n\times f}$ such that $\mathbf{M} = \mathbf{X}\mathbf{X}^\intercal$.\\[1em]

Given the non-thin Single-Value Decomposition: $\mathbf{X} = \mathbf{U}\boldsymbol\Sigma\mathbf{V}^\intercal$, we can write $\mathbf{M}$ as $\mathbf{U}\boldsymbol\Sigma\boldsymbol\Sigma^\intercal\mathbf{U}^\intercal$. Then we see that $\mathbf{M}$ is the Gram matrix of the rows of $\mathbf{U}\boldsymbol\Sigma = \overset{\sim}{\mathbf{X}}$, which we may use as our embedding.


\section{Question 3}\label{sec:q-3}
\subsection{Part (a)}

Degradation\cite{DBLP:journals/corr/HeZRS15}, as it pertains to feed forward neural networks refers to the phenomenon whereby as the depth of the network increases, both the training and test accuracy saturate (change very little with increasing depth), and then start to get worse.\\[1em]

This runs counter to expectation: As the depth of a network increases, the number of parameters in the model also increases, and typically, this causes a model to \textbf{overfit}. This would be characterised by the training accuracy constantly improving with depth until it reaches 100\% whilst after a point, the test accuracy starts to suffer. This is different from degradation in which both accuracies suffer.\\[1em]

In the case of linear regression with a polynomial basis expansion, I would expect overfit to occur, and not degradation: It is always possible to fit a polynomial of degree $\geq n$ through $n$ points, and linear regression will always find such a fit (if regularisation techniques are not employed), so given $n$ sample points, as the degree $d$ of the basis expansion approaches $n$ from below, I would expect training accuracy to approach 100\% (and stay at 100\% as $d$ exceeds $n$). Meanwhile, test accuracy will deterioriate as this model will generalise poorly.\\[1em]

The reason positted for feed-forward neural networks' susceptibility to degradation is that optimisation techniques may have trouble approximating identity mappings using multiple layers of non-linearities. Such a difficulty would mean that even if there is a $d$-layer neywork, $N$, with better training accuracy than a given $d+k$ layer network ($k > 0$), $N^\prime$, by optimising the latter, we are unlikely to reach a network where the first $d$ layers resemble the former, and the last $k$ layers are identity mappings (even though this is an improvement), or indeed any other interleaving of layers in $N$ and identity mappings.

\subsection{Part (b)}

Both Residual and Highway networks\cite{DBLP:journals/corr/SrivastavaGS15a} address issues that arise when attempting to train very deep feed-forward networks, by designing networks where the input to a layer can skip it (via a shortcircuit, or an ``information highway'' respectively) and be sent through to later layers.\\[1em]

Given a plain 4-layer network with $\operatorname{ReLU}$ non-linearities (biases omitted for brevity):
\begin{align*}
  &\operatorname{ReLU} : \mathbb{R}^n\to\mathbb{R}^n\\
  &\operatorname{ReLU}(\mathbf{x}) = \mathbf{x}^\prime\text{ where }x^\prime_i = \max(0,x_i)
\end{align*}
\begin{align*}
  &\text{Weights}\quad&\mathbf{W}_i & \in\mathbb{R}^{n\times n} && \text{for}\quad i = 1,\ldots,4\\
  &&\mathbf{z}_i & \in\mathbb{R} &&\text{for}\quad i = 0,\ldots,4\\
  &&\mathbf{z}_{i+1} & = \operatorname{ReLU}(\mathbf{W}_{i+1}\mathbf{z}_i) &&\text{for}\quad i = 0,\ldots,3\\
  &\textbf{input}\quad&\mathbf{x} & = \mathbf{z}_0\\
  &\textbf{output}\quad&\mathbf{y} & = \mathbf{z}_4
\end{align*}

We may make it a residual network by adding shortcircuits every 2 layers:\marginnote{The choice of 2 layers was arbitrary for the sake of the example, both techniques work when skipping any number of layers, although they must skip over a non-linearity in order to change the behaviour of the network.}

\begin{align*}
  \mathbf{z}_{i+1} & =
  \begin{cases}
    \operatorname{ReLU}(\mathbf{W}_{i+1}\mathbf{z}_i + \mathbf{z}_{i-1})
    & \text{ if $i+1$ is even}\\
    \operatorname{ReLU}(\mathbf{W}_{i+1}\mathbf{z}_i)
    & \text{ otherwise}
  \end{cases}
  \intertext{Or we may make it a highway network by adding a transfer and carry gate after every 2 layers:}
  \mathbf{z}_{i+1} & =
  \begin{cases}
    \operatorname{ReLU}(\mathbf{W}_{i+1}\mathbf{z}_i)\odot T(\mathbf{z}_{i-1},\mathbf{U}_{i+1})
    +\mathbf{z}_{i-1}\odot C(\mathbf{z}_{i-1},\mathbf{V}_{i+1})
    & \text{ if $i+1$ is even}\\
    \operatorname{ReLU}(\mathbf{W}_{i+1}\mathbf{z}_i)
    & \text{ otherwise}
  \end{cases}
\end{align*}

Where $T$ and $C$ define the transform and carry gates respectively and $\odot$ denotes elementwise multiplication.\\[1em]

As can be seen, the ``skipping'' is not true skipping in either case. Instead, the input to a layer is mixed with the output of a later layer.\\[1em]

Residual networks could be considered a special case of Highway networks, where the transfer and carry gates are both always open. As a result, short-circuits in Residual networks add no extra parameters whereas those in Highway networks do ($\mathbf{U}$ and $\mathbf{V}$ in the example).\\[1em]

\subsection{Part (c)}

Lesioning is the process of essentially ``disabling'' a layer: Its transfer gate is manually set to $\mathbf{0}$ and (as the carry gate is defined by $\mathbf{1}-\mathbf{T}$ in the paper) its carry gate becomese $\mathbf{1}$, causing it to just copy its input to its output.

It was used to see which layers are significant to the output of the network. The conclusion was that in most networks, lesioning early layers was most detrimental, with later layers becoming progressively less significant. What is more, for networks for simpler learning problems, like MNIST (Handwritten digits), later layers contributed almost nothing to the output, but an equally deep highway network used for a more complex learning task, such as CIFAR-100 (Object recognition, 100 classes), utilised all of its layers. The method used to determine this was, after training, a given layer was lesioned, the network was run on the training set again and the difference in the error was measured. A large difference suggested the layer contributed to the network, and a small difference in error suggested the opposite.\\[1em]

This indicates that Highway networks are capable of scaling he number of layers it uses based on the needs of the specific classification task at hand, which is a desirable trait but one that (as hinted in the first paper) is not easy to achieve with plain feed forward networks.


\end{document}
